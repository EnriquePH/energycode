\documentclass[12pt]{article}
\usepackage{amsmath, amssymb}
\usepackage[utf8]{inputenc}
\usepackage[spanish]{babel}
\usepackage{geometry}
\usepackage{hyperref}
\geometry{margin=2.5cm}

\title{Máximo Común Divisor entre términos de Fibonacci}
\author{}
\date{}

\begin{document}

\maketitle

\section*{Teorema}

Para todos \( m, n \in \mathbb{N} \), se cumple que:
\[
\gcd(F_m, F_n) = F_{\gcd(m, n)}
\]

Es decir, el máximo común divisor de dos términos de la sucesión de Fibonacci es igual al término de Fibonacci cuyo índice es el máximo común divisor de los índices originales.

\section*{Demostración}

Sea la sucesión de Fibonacci definida por:
\[
F_0 = 0, \quad F_1 = 1, \quad F_{n+1} = F_n + F_{n-1}, \quad \text{para todo } n \geq 1.
\]

Queremos demostrar que para todo \( m, n \in \mathbb{N} \):
\[
\gcd(F_m, F_n) = F_{\gcd(m, n)}.
\]

La demostración formal puede hacerse por inducción o usando el algoritmo de Euclides aplicado a índices, y se basa en la compatibilidad de la sucesión de Fibonacci con las operaciones de MCD.

Aplicamos el algoritmo de Euclides sobre los índices:

\[
\gcd(m, n) = \gcd(n, m \bmod n)
\]

Luego, por una propiedad recurrente de la sucesión de Fibonacci, se puede demostrar que:

\[
\gcd(F_m, F_n) = \gcd(F_n, F_{m \bmod n})
\]

Repitiendo el proceso, se llega a:

\[
\gcd(F_m, F_n) = F_{\gcd(m, n)}
\]

Este resultado ha sido demostrado formalmente y aparece en diversas fuentes de teoría de números.

\hfill\(\blacksquare\)

\section*{Corolario}

Para \( k > 1 \), se sigue que:
\[
\gcd(F_{n+k}, F_n) = F_{\gcd(n, k)}
\]

\section*{Ejemplos}

\begin{itemize}
    \item Si \( n = 10 \) y \( k = 4 \), entonces \( \gcd(10, 4) = 2 \), y:
    \[
    \gcd(F_{14}, F_{10}) = F_2 = 1
    \]
    \item Si \( n = 12 \), \( k = 6 \), entonces \( \gcd(12, 6) = 6 \), y:
    \[
    \gcd(F_{18}, F_{12}) = F_6 = 8
    \]
    \item Si \( m = 16 \) y \( n = 12 \), entonces \( \gcd(16, 12) = 4 \), y:
    \[
    \gcd(F_{16}, F_{12}) = F_4 = 3
    \]
\end{itemize}

\section*{Bibliografía}

\begin{enumerate}
    \item Koshy, T. (2001). \textit{Fibonacci and Lucas Numbers with Applications}. Wiley-Interscience.
    Capítulo 5 contiene la demostración y aplicaciones del teorema \( \gcd(F_m, F_n) = F_{\gcd(m, n)} \).

    \item Dunlap, R. A. (1997). \textit{The Golden Ratio and Fibonacci Numbers}. World Scientific Publishing.
    Discusión del MCD entre números de Fibonacci en el contexto de propiedades algebraicas.

    \item Niven, I., Zuckerman, H. S., \& Montgomery, H. L. (1991). \textit{An Introduction to the Theory of Numbers} (5ª ed.). Wiley.
    Contiene secciones sobre sucesiones recurrentes y divisibilidad.

    \item Rosen, K. H. (2011). \textit{Elementary Number Theory and Its Applications} (6ª ed.). Pearson.
    Incluye ejercicios y teoremas relacionados con la sucesión de Fibonacci y divisibilidad.

\end{enumerate}

\end{document}


