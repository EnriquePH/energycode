% Options for packages loaded elsewhere
\PassOptionsToPackage{unicode}{hyperref}
\PassOptionsToPackage{hyphens}{url}
\PassOptionsToPackage{dvipsnames,svgnames,x11names}{xcolor}
%
\documentclass[
  12pt,
  letterpaper,
  DIV=11,
  numbers=noendperiod]{scrartcl}

\usepackage{amsmath,amssymb}
\usepackage{iftex}
\ifPDFTeX
  \usepackage[T1]{fontenc}
  \usepackage[utf8]{inputenc}
  \usepackage{textcomp} % provide euro and other symbols
\else % if luatex or xetex
  \usepackage{unicode-math}
  \defaultfontfeatures{Scale=MatchLowercase}
  \defaultfontfeatures[\rmfamily]{Ligatures=TeX,Scale=1}
\fi
\usepackage{lmodern}
\ifPDFTeX\else  
    % xetex/luatex font selection
\fi
% Use upquote if available, for straight quotes in verbatim environments
\IfFileExists{upquote.sty}{\usepackage{upquote}}{}
\IfFileExists{microtype.sty}{% use microtype if available
  \usepackage[]{microtype}
  \UseMicrotypeSet[protrusion]{basicmath} % disable protrusion for tt fonts
}{}
\makeatletter
\@ifundefined{KOMAClassName}{% if non-KOMA class
  \IfFileExists{parskip.sty}{%
    \usepackage{parskip}
  }{% else
    \setlength{\parindent}{0pt}
    \setlength{\parskip}{6pt plus 2pt minus 1pt}}
}{% if KOMA class
  \KOMAoptions{parskip=half}}
\makeatother
\usepackage{xcolor}
\usepackage[margin=1in]{geometry}
\setlength{\emergencystretch}{3em} % prevent overfull lines
\setcounter{secnumdepth}{-\maxdimen} % remove section numbering
% Make \paragraph and \subparagraph free-standing
\makeatletter
\ifx\paragraph\undefined\else
  \let\oldparagraph\paragraph
  \renewcommand{\paragraph}{
    \@ifstar
      \xxxParagraphStar
      \xxxParagraphNoStar
  }
  \newcommand{\xxxParagraphStar}[1]{\oldparagraph*{#1}\mbox{}}
  \newcommand{\xxxParagraphNoStar}[1]{\oldparagraph{#1}\mbox{}}
\fi
\ifx\subparagraph\undefined\else
  \let\oldsubparagraph\subparagraph
  \renewcommand{\subparagraph}{
    \@ifstar
      \xxxSubParagraphStar
      \xxxSubParagraphNoStar
  }
  \newcommand{\xxxSubParagraphStar}[1]{\oldsubparagraph*{#1}\mbox{}}
  \newcommand{\xxxSubParagraphNoStar}[1]{\oldsubparagraph{#1}\mbox{}}
\fi
\makeatother


\providecommand{\tightlist}{%
  \setlength{\itemsep}{0pt}\setlength{\parskip}{0pt}}\usepackage{longtable,booktabs,array}
\usepackage{calc} % for calculating minipage widths
% Correct order of tables after \paragraph or \subparagraph
\usepackage{etoolbox}
\makeatletter
\patchcmd\longtable{\par}{\if@noskipsec\mbox{}\fi\par}{}{}
\makeatother
% Allow footnotes in longtable head/foot
\IfFileExists{footnotehyper.sty}{\usepackage{footnotehyper}}{\usepackage{footnote}}
\makesavenoteenv{longtable}
\usepackage{graphicx}
\makeatletter
\newsavebox\pandoc@box
\newcommand*\pandocbounded[1]{% scales image to fit in text height/width
  \sbox\pandoc@box{#1}%
  \Gscale@div\@tempa{\textheight}{\dimexpr\ht\pandoc@box+\dp\pandoc@box\relax}%
  \Gscale@div\@tempb{\linewidth}{\wd\pandoc@box}%
  \ifdim\@tempb\p@<\@tempa\p@\let\@tempa\@tempb\fi% select the smaller of both
  \ifdim\@tempa\p@<\p@\scalebox{\@tempa}{\usebox\pandoc@box}%
  \else\usebox{\pandoc@box}%
  \fi%
}
% Set default figure placement to htbp
\def\fps@figure{htbp}
\makeatother

\KOMAoption{captions}{tableheading}
\makeatletter
\@ifpackageloaded{caption}{}{\usepackage{caption}}
\AtBeginDocument{%
\ifdefined\contentsname
  \renewcommand*\contentsname{Table of contents}
\else
  \newcommand\contentsname{Table of contents}
\fi
\ifdefined\listfigurename
  \renewcommand*\listfigurename{List of Figures}
\else
  \newcommand\listfigurename{List of Figures}
\fi
\ifdefined\listtablename
  \renewcommand*\listtablename{List of Tables}
\else
  \newcommand\listtablename{List of Tables}
\fi
\ifdefined\figurename
  \renewcommand*\figurename{Figure}
\else
  \newcommand\figurename{Figure}
\fi
\ifdefined\tablename
  \renewcommand*\tablename{Table}
\else
  \newcommand\tablename{Table}
\fi
}
\@ifpackageloaded{float}{}{\usepackage{float}}
\floatstyle{ruled}
\@ifundefined{c@chapter}{\newfloat{codelisting}{h}{lop}}{\newfloat{codelisting}{h}{lop}[chapter]}
\floatname{codelisting}{Listing}
\newcommand*\listoflistings{\listof{codelisting}{List of Listings}}
\makeatother
\makeatletter
\makeatother
\makeatletter
\@ifpackageloaded{caption}{}{\usepackage{caption}}
\@ifpackageloaded{subcaption}{}{\usepackage{subcaption}}
\makeatother

\usepackage{bookmark}

\IfFileExists{xurl.sty}{\usepackage{xurl}}{} % add URL line breaks if available
\urlstyle{same} % disable monospaced font for URLs
\hypersetup{
  pdftitle={Análisis de Matrices Escalonadas: Potencias, Determinantes y Normas},
  pdfauthor={Inteligencia Artificial Asistida},
  colorlinks=true,
  linkcolor={blue},
  filecolor={Maroon},
  citecolor={Blue},
  urlcolor={Blue},
  pdfcreator={LaTeX via pandoc}}


\title{Análisis de Matrices Escalonadas: Potencias, Determinantes y
Normas}
\author{Inteligencia Artificial Asistida}
\date{}

\begin{document}
\maketitle


\section{Resumen}\label{resumen}

Este artículo analiza cuatro matrices cuadradas de dimensión
\(n \times n\) con estructuras específicas sobre o bajo la diagonal
principal y la diagonal secundaria. Se estudia su comportamiento al
elevarlas a una potencia \(k\), sus determinantes y varias normas
matriciales. Se incluyen ejemplos concretos y aplicaciones en
matemáticas y física.

\section{Definición de las
matrices}\label{definiciuxf3n-de-las-matrices}

Dado un entero \(n \geq 1\), se definen las siguientes matrices:

\begin{itemize}
\item
  \textbf{Matriz A} (triangular superior con unos): {[} A\_\{i,j\} =

  \begin{cases}
      1 & \text{si } i \le j \\
      0 & \text{si } i > j
  \end{cases}

  {]}
\item
  \textbf{Matriz B} (triangular inferior con unos): {[} B\_\{i,j\} =

  \begin{cases}
      1 & \text{si } i \ge j \\
      0 & \text{si } i < j
  \end{cases}

  {]}
\item
  \textbf{Matriz C} (diagonal secundaria y superiores con unos): {[}
  C\_\{i,j\} =

  \begin{cases}
      1 & \text{si } i + j \le n - 1 \\
      0 & \text{si } i + j > n - 1
  \end{cases}

  {]}
\item
  \textbf{Matriz D} (diagonal secundaria e inferiores con unos): {[}
  D\_\{i,j\} =

  \begin{cases}
      1 & \text{si } i + j \ge n - 1 \\
      0 & \text{si } i + j < n - 1
  \end{cases}

  {]}
\end{itemize}

\section{\texorpdfstring{Potencias de las matrices
\(M^k\)}{Potencias de las matrices M\^{}k}}\label{potencias-de-las-matrices-mk}

\subsection{Matriz A}\label{matriz-a}

\(A\) es una matriz unipotente triangular superior. Sus potencias están
dadas por: {[} A\^{}k = \sum\emph{\{m=0\}\^{}\{k\} \binom{k}{m} N\^{}m,
\quad \text{donde } N = A - I {]} {[} (A\^{}k)}\{i,j\} =
\binom{k}{j - i}, \quad \text{si } j \ge i {]}

\subsection{Matriz B}\label{matriz-b}

Simétricamente: {[} (B\^{}k)\_\{i,j\} = \binom{k}{i - j},
\quad \text{si } i \ge j {]}

\subsection{Matriz C y D}\label{matriz-c-y-d}

Ambas se obtienen como reflexión respecto a la diagonal secundaria: {[}
C = JAJ, \quad D = JBJ \quad \Rightarrow \quad C\^{}k = JA\^{}kJ,
\quad D\^{}k = JB\^{}kJ {]} donde \(J\) es la matriz de reverso de
orden.

\section{\texorpdfstring{Determinantes de
\(M^k\)}{Determinantes de M\^{}k}}\label{determinantes-de-mk}

\begin{itemize}
\tightlist
\item
  \(\det(A^k) = \det(B^k) = 1\), por ser triangulares unipotentes.
\item
  \(\det(C^k) = \det(D^k) = 0\), ya que tienen filas linealmente
  dependientes o ceros en la diagonal secundaria.
\end{itemize}

\section{\texorpdfstring{Normas Matriciales de
\(M^k\)}{Normas Matriciales de M\^{}k}}\label{normas-matriciales-de-mk}

\begin{itemize}
\tightlist
\item
  \textbf{Norma 1:} \(\|M^k\|_1 = \max_j \sum_i |(M^k)_{i,j}|\)
\item
  \textbf{Norma \(\infty\):}
  \(\|M^k\|_\infty = \max_i \sum_j |(M^k)_{i,j}|\)
\item
  \textbf{Norma de Frobenius:}
  \(\|M^k\|_F = \sqrt{ \sum_{i,j} (M^k)_{i,j}^2 }\)
\end{itemize}

Para \(A^k\) y \(B^k\), estas normas se expresan mediante combinatorias:
{[} \textbar A\textsuperscript{k\textbar{}\emph{1 =
\textbar B\textsuperscript{k\textbar{}\emph{\infty =
\sum}\{m=0\}}\{\min(k, n-1)\} \binom{k}{m} {]} {[}
\textbar A\textsuperscript{k\textbar{}\emph{F\^{}2 =
\sum}\{i=0\}}\{n-1\} \sum}\{j=i\}}\{n-1\} \binom{k}{j-i}\^{}2 {]}

\(C^k\) y \(D^k\) tienen las mismas normas que \(A^k\) y \(B^k\)
respectivamente, por simetría.

\section{\texorpdfstring{Ejemplos con
\(n = 5\)}{Ejemplos con n = 5}}\label{ejemplos-con-n-5}

\subsection{Matriz A}\label{matriz-a-1}

{[}

\begin{bmatrix}
1 & 1 & 1 & 1 & 1 \\
0 & 1 & 1 & 1 & 1 \\
0 & 0 & 1 & 1 & 1 \\
0 & 0 & 0 & 1 & 1 \\
0 & 0 & 0 & 0 & 1 \\
\end{bmatrix}

{]}

\subsection{Matriz B}\label{matriz-b-1}

{[}

\begin{bmatrix}
1 & 0 & 0 & 0 & 0 \\
1 & 1 & 0 & 0 & 0 \\
1 & 1 & 1 & 0 & 0 \\
1 & 1 & 1 & 1 & 0 \\
1 & 1 & 1 & 1 & 1 \\
\end{bmatrix}

{]}

\subsection{Matriz C}\label{matriz-c}

{[}

\begin{bmatrix}
1 & 1 & 1 & 1 & 1 \\
1 & 1 & 1 & 1 & 0 \\
1 & 1 & 1 & 0 & 0 \\
1 & 1 & 0 & 0 & 0 \\
1 & 0 & 0 & 0 & 0 \\
\end{bmatrix}

{]}

\subsection{Matriz D}\label{matriz-d}

{[}

\begin{bmatrix}
0 & 0 & 0 & 0 & 1 \\
0 & 0 & 0 & 1 & 1 \\
0 & 0 & 1 & 1 & 1 \\
0 & 1 & 1 & 1 & 1 \\
1 & 1 & 1 & 1 & 1 \\
\end{bmatrix}

{]}

\section{Aplicaciones}\label{aplicaciones}

\begin{itemize}
\tightlist
\item
  \textbf{Álgebra lineal computacional:} resolución eficiente de
  sistemas por sustitución.
\item
  \textbf{Modelos temporales:} propagación acumulativa (memoria), series
  autoregresivas.
\item
  \textbf{Teoría de grafos:} conteo de caminos jerárquicos o cruzados.
\item
  \textbf{Física computacional:} flujos unidireccionales, redes de
  difusión, operadores diferenciales discretizados.
\item
  \textbf{Criptografía:} codificación lineal y matrices de
  transformación escalonadas.
\end{itemize}

\section{Conclusión}\label{conclusiuxf3n}

Las matrices con estructuras triangulares o antitriangulares son
fundamentales en múltiples disciplinas. Su comportamiento bajo potencias
revela patrones combinatorios precisos. Mientras que \(A\) y \(B\)
conservan determinantes unitarios, \(C\) y \(D\) pierden rango. Las
normas crecen exponencialmente según combinaciones binomiales,
reflejando acumulación jerárquica. Estas propiedades las hacen
esenciales en teoría, aplicaciones y cómputo.

\section{Referencias}\label{referencias}

\begin{enumerate}
\def\labelenumi{\arabic{enumi}.}
\tightlist
\item
  Horn, R. A., \& Johnson, C. R. (2012). \emph{Matrix Analysis}.
  Cambridge University Press.
\item
  Meyer, C. D. (2000). \emph{Matrix Analysis and Applied Linear
  Algebra}. SIAM.
\item
  Strang, G. (2016). \emph{Introduction to Linear Algebra}.
  Wellesley-Cambridge Press.
\item
  Lay, D. C. et al.~(2016). \emph{Linear Algebra and Its Applications}.
  Pearson.
\end{enumerate}




\end{document}
